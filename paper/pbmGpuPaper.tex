\documentclass[preprint,11pt,authoryear]{elsarticle}

%% Use the option review to obtain double line spacing
%% \documentclass[preprint,review,12pt]{elsarticle}

%% Use the options 1p,twocolumn; 3p; 3p,twocolumn; 5p; or 5p,twocolumn
%% for a journal layout:
%% \documentclass[final,1p,times]{elsarticle}
%% \documentclass[final,1p,times,twocolumn]{elsarticle}
%% \documentclass[final,3p,times]{elsarticle}
%% \documentclass[final,3p,times,twocolumn]{elsarticle}
%% \documentclass[final,5p,times]{elsarticle}
%% \documentclass[final,5p,times,twocolumn]{elsarticle}

%% if you use PostScript figures in your article

%% or use the graphicx package for more complicated commands
\usepackage{graphicx}
%% or use the epsfig package if you prefer to use the old commands
\usepackage{epsfig}

%% The amssymb package provides various useful mathematical symbols
\usepackage{amssymb}
%% The amsthm package provides extended theorem environments
%% \usepackage{amsthm}

%% The lineno packages adds line numbers. Start line numbering with
%% \begin{linenumbers}, end it with \end{linenumbers}. Or switch it on
%% for the whole article with \linenumbers after \end{frontmatter}.
\usepackage{lineno}

%% natbib.sty is loaded by default. However, natbib options can be
%% provided with \biboptions{...} command. Following options are
%% valid:

%%   round  -  round parentheses are used (default)
%%   square -  square brackets are used   [option]
%%   curly  -  curly braces are used      {option}
%%   angle  -  angle brackets are used    <option>
%%   semicolon  -  multiple citations separated by semi-colon
%%   colon  - same as semicolon, an earlier confusion
%%   comma  -  separated by comma
%%   numbers-  selects numerical citations
%%   super  -  numerical citations as superscripts
%%   sort   -  sorts multiple citations according to order in ref. list
%%   sort&compress   -  like sort, but also compresses numerical citations
%%   compress - compresses without sorting
%%
%% \biboptions{comma,round}

% \biboptions{}
\usepackage{hyperref}
\usepackage{color}
\usepackage{textcomp}
\usepackage{multirow}
\usepackage{float}

\journal{Computer and Chemical Engineering}

\begin{document}

\begin{frontmatter}
\title{Accelerating multi-dimensional population balance model simulations using GPU with a highly scalable framework.}


\author[label1]{Chaitanya Sampat}
\author[label1]{Yukteshwar Baranwal}
\address[label1]{Chemical and Biochemical Engineering, Rutgers University, Piscataway, NJ, USA - 08854}
\author[label1]{Rohit Ramachandran\corref{cor1}}
\ead{rohitrr@soemail.rutgers.com}
\cortext[cor1]{Corresponding author}
%\ead[url]{pslrutgers.com}

\begin{abstract}
Population Balance models are widely used in pharmaceutical industry to simulate
and optimize the granulation process.Prediction of the distribution of many 
particulate sytems with the evolution of time using these models, has become an 
efficient tool for the on-line control of the granulation process \citep{Prakash2013a}. 
With increase in the number of components and phases in a particulate mixture, 
complexity of PBM increases. This leads to multi-dimensional matrix calculations 
requiring great computational power. Over the past decade, graphical process 
units (GPUs) are increasingly being used for computation. This study focuses on the 
development of an algorithm to parallelize the various nested loops inside the PBM. 
The PBM was developed for a continuous high shear granulation process and was made 
specifically for NVIDIA GPUs as it was developed on CUDA C/C++. The communication 
time is much lower in comparison to the speedup achieved in the parallel section 
of the code on the GPUs. The speed up achieved was significant compared to the PBM 
code when run in serial or in on multi-core configuration. The speed improvements 
for the code was reported for various CPU \& GPU architectures and configurations. 
The speed improvements were also tested with various combinations in 
discretization of the granulator geometry and threads used for calculations. 
\end{abstract}

\begin{keyword}
%% keywords here, in the form: keyword \sep keyword
Population Balance Model \sep GPU \sep Paralllel Computing \sep Granulation \sep MPI
%% MSC codes here, in the form: \MSC code \sep code
%% or \MSC[2008] code \sep code (2000 is the default)
\end{keyword}

\end{frontmatter}

\begin{linenumbers}
%% main text
\section{Introduction}
\label{secIntro}
Various chemical industries like detergent, food, pharmacuetical, fertilizers, 
catalyst, etc deal with particulate processes. They constitute to about 50\% 
of the world's chemical production \citep{seville1997}. In the pharmacuetical 
industry, granulations plays an important role in the tablet 
manufacturing. Granulation is the process in which fine pharmacuetical powder 
blends are converted to larger granules using a liquid or a dry binder \citep{Chaturbedi2017}. 
These larger granules help in better flow ability and strength to these mixtures 
aiding further processing. 

%Due to the large number of collisions inside these systems, modeling these systems is a challenge.

Understanding the dynamics of the continuous granulating system is vital for its smooth 
operation and to reduce the amount of waste generated in the development phase. The Food 
and Drug administration (FDA) has also been promoting a similar initiative with its 
Quality by Design (QbD) and Process Analytical Tools (PAT) principles~\citep{sen2014}. 
Thus, a process model for the system becomes an integral part of the development phase. 
A model which predicts the bulk mechanical properties of the mixture as well as a particle size 
distribution (PSD) is required. Over the past decade, Population Balance Models (PBMs) have been used 
to predict the behaviour of granulation processes \citep{Barrasso2013},\citep{Ramachandran2009}. 

PBMs are used to calculate bulk rate processes ocurring during granulation. PBMs are sometimes 
unable to integrate certain information, thus a mechanistic kernel can be 
introduced in these models to make them more accurate. Another way to increase its accuracy, 
is to incorporate larger number of solid bins inside the PBM. The increase in the number of 
solid bins leads to an increase in calculations for each time step, leading to a higher 
simulation times. The calculations increase by a factor of $2^n$ where n being the number 
of solid bins. An accurate model which incorporates higher number of solid bins as well as 
includes a mechanistic kernel in its calculations is expected to be very 
slow to simulate and could take upto hours to complete. Such models and their solving 
techniques are not viable to be used in real time sytem control. Thus, there is a need to 
solve these models quicker.

The advancement of computers and its peripherals in recent years have led to a great increase in 
computational resources leading to faster simulations. The recent central processing unit (CPU) 
now contains various of cores thus making it possible to run multiples processes in parallel. 
In order to take advantege of a highly parallel framework, large number of cores are needed 
which may not be possible in a personal desktop and a supercomputer cluster needs to be used.
Another computer peripheral that can to be used to run a highly parallel code is the computer's 
graphic processing unit (GPU)~\citep{Prakash2013b}. These GPUs contain thousands of compute 
cores that can be used run tasks in parallel. Thus, a desktop equipped with a GPU could 
compute the same results as a CPU code on supercomputers in lesser amount of time as seen in Section \ref{secResults}.
With the launch of Compute Unified Device Architecture (CUDA), NVIDIA made it easier to use GPUs for 
general parallel programming in an approach usually termed as general purpose computing on GPUs (GPGPUs).

In the present study, a mechanistic multi-dimensional PBM was developed such that it was not 
only accurate but also scalable since the number of solid bins could be changed to alter its 
behaviour. This model was developed in C++ to be run on CPUs. This C++ model was parallelised 
using Message Parsing Interace (MPI) which is parallel application programming interface (API). 
This model was developed NVIDIA GPUs and was parallelized using the CUDA toolkit. The timings of 
the simulations were then compared for each of these cases. The scalability of the GPU based code 
was also tested to obtain speed improvements over serial CPU code.














Introduce granulation \\
Start with the importance of granulation in pharmacuetical processes \\
Talk about the how particles sizes is important and can be predicted using PBM \\
introduce PBM \\ 
multi-dimensional PBM is very slow due to added calculations \\ 
thus parallel computing is needed for faster simulations \\ 
talk about development of GPUs \\ 
Introduce work for GPUs and its importance

\section{Background and motivation}
\label{secBkgd}
\subsection{Population Balance Model}
talk about all the processes involved in granulation \\
\subsection{Parallel Computing}
Talk about general computer architecture\\
then move onto cpus and cores \\
introduce mpi and its advantages 
\subsection{GPU computing Basics}
Talk about GPUs\\
advantages like large number of cores, lesser power consumption etc\\
Talk about CUDA toolkit and some other basics \\ 
Explain task parallelisation and data parallelisation \\
talk about threads blocks and grids etc

\subsection{Previous parallel PBM works}
cite cdse, Santos, Gunuwan and other from CDSE
cite Nagy, Santos, Green etc for the PBM GPU part

\section{Method and implementation}
\label{secMethods}
\subsection{PBM implementation}
relate each equation of the PBM used to each process and explain a bit
\subsection{MPI implementation}
elaborate how the geometry was divided into compartments and how each \\
compartment was sent to each block
\subsection{GPU implementation}
Elaborate how each block was each compartment and how each thread was 
parallelized to perform calculation for solid 1 solid 2 combination.

\section{Results and discussions}
\label{secResults}

\section{Conclusions}
\label{secConc}
\end{linenumbers}

\bibliographystyle{elsarticle-harv}
\bibliography{pbmGpuPaper}


\end{document}

%%
%% End of file `elsarticle-template-num.tex'.
