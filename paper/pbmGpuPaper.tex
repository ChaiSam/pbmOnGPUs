\documentclass[preprint,12pt,authoryear]{elsarticle}

%% Use the option review to obtain double line spacing
%% \documentclass[preprint,review,12pt]{elsarticle}

%% Use the options 1p,twocolumn; 3p; 3p,twocolumn; 5p; or 5p,twocolumn
%% for a journal layout:
%% \documentclass[final,1p,times]{elsarticle}
%% \documentclass[final,1p,times,twocolumn]{elsarticle}
%% \documentclass[final,3p,times]{elsarticle}
%% \documentclass[final,3p,times,twocolumn]{elsarticle}
%% \documentclass[final,5p,times]{elsarticle}
%% \documentclass[final,5p,times,twocolumn]{elsarticle}

%% if you use PostScript figures in your article
%% use the graphics package for simple commands
%% \usepackage{graphics}
%% or use the graphicx package for more complicated commands
%% \usepackage{graphicx}
%% or use the epsfig package if you prefer to use the old commands
%% \usepackage{epsfig}

%% The amssymb package provides various useful mathematical symbols
\usepackage{amssymb}
%% The amsthm package provides extended theorem environments
%% \usepackage{amsthm}

%% The lineno packages adds line numbers. Start line numbering with
%% \begin{linenumbers}, end it with \end{linenumbers}. Or switch it on
%% for the whole article with \linenumbers after \end{frontmatter}.
\usepackage{lineno}

%% natbib.sty is loaded by default. However, natbib options can be
%% provided with \biboptions{...} command. Following options are
%% valid:

%%   round  -  round parentheses are used (default)
%%   square -  square brackets are used   [option]
%%   curly  -  curly braces are used      {option}
%%   angle  -  angle brackets are used    <option>
%%   semicolon  -  multiple citations separated by semi-colon
%%   colon  - same as semicolon, an earlier confusion
%%   comma  -  separated by comma
%%   numbers-  selects numerical citations
%%   super  -  numerical citations as superscripts
%%   sort   -  sorts multiple citations according to order in ref. list
%%   sort&compress   -  like sort, but also compresses numerical citations
%%   compress - compresses without sorting
%%
%% \biboptions{comma,round}

% \biboptions{}
\usepackage{hyperref}
\usepackage{color}
\usepackage{textcomp}
\usepackage{multirow}
\usepackage{float}

\journal{Some Journal}

\begin{document}

\begin{frontmatter}

%\title{Sample article to present \texttt{elsarticle} class\tnoteref{label0}}
%\tnotetext[label0]{This is only an example}

\title{Title of Paper}


\author[label1,label2]{Author One\corref{cor1}\fnref{label3}}
\address[label1]{Address One}
\address[label2]{Address Two\fnref{label4}}

\cortext[cor1]{I am corresponding author}
\fntext[label3]{I also want to inform about\ldots}
%\fntext[label4]{Small city}

\ead{author.one@mail.com}
\ead[url]{author-one-homepage.com}

\author[label5]{Author Two}
\address[label5]{Some University}
\ead{author.two@mail.com}

\author[label1,label5]{Author Three}
\ead{author.three@mail.com}

\begin{abstract}
Text of abstract. Text of abstract. Text of abstract. Text of abstract. Text of abstract. 
\end{abstract}

\begin{keyword}
%% keywords here, in the form: keyword \sep keyword
Keyword 1 \sep Keyword 2 \sep Keyword 3 \sep Keyword 4 \sep Keyword 5
%% MSC codes here, in the form: \MSC code \sep code
%% or \MSC[2008] code \sep code (2000 is the default)
\end{keyword}

\end{frontmatter}

%%
%% Start line numbering here if you want
%%
\linenumbers

%% main text
\section{Introduction}
\label{secIntro}



\bibliography{pbmGpuPaper}


\end{document}

%%
%% End of file `elsarticle-template-num.tex'.
